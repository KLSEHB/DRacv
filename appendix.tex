\documentclass[10pt,journal,compsoc]{IEEEtran}
\IEEEoverridecommandlockouts
% The preceding line is only needed to identify funding in the first footnote. If that is unneeded, please comment it out.
\usepackage{xcolor}
\usepackage{multirow}
\usepackage{booktabs}
\usepackage{amsmath}
\usepackage[noend]{algpseudocode}
\usepackage{algorithmicx,algorithm}
\usepackage{CJKutf8}
\usepackage{multirow}
\usepackage{graphicx} 
\usepackage{subcaption}
\usepackage{comment}
\usepackage{ulem}
\usepackage{makecell}
\usepackage{algorithm,algpseudocode,amsmath}
\usepackage[section]{placeins}
\usepackage{float}

\begin{document}
\begin{CJK}{UTF8}{gbsn}
\appendix
\section{Patch codes of other nine open-source Web applications}
\label{appendix:Repair code of other nine open-source Web applications}

\textbf{AWCMs} is a web resource management application that encompasses categories such as videos, themes, sounds, and photos. Strict user verification can be enforced by using the privilege parameter \textit{username} to repair BAC vulnerabilities in \textit{VPs}, such as \textit{member.php}. Additionally, the role's privilege verification can be employed to repair BAC vulnerabilities in \textit{VPs} including \textit{m\_cp\_avatar.php, member.php, install/index.php}, and \textit{db\_backup.php} by utilizing the privilege parameter $awcm\_cp$. If the value of the \textit{awcm\_cp} parameter is set to "\textit{no}", the application will redirect the user to the \textit{index.php} page. As shown in Fig.\ref{AWCMs}.

\textbf{PHPOLL} is a simple voting web application. Its privilege parameters are extracted as \textit{string\_cook\_login} and \textit{string\_cook\_password}. They are used in the \textit{VP} \textit{modifica\_votanti.php, modifica\_band, and modifica\_configurazione} to repair the vulnerability. As shown in Fig. \ref{Phpoll}.

\textbf{Bwapp} is a vulnerability demonstration platform that utilizes privilege parameters, such as \textit{username} and \textit{password}, to ensure strict user verification and address vulnerabilities in the \textit{VPs} \textit{backdoor.php} and \textit{install.php}. Additionally, the platform employs the privilege parameter \textit{security\_level} to implement security level distinction, namely, role's privilege verification, on these pages. The \textit{security\_level} parameter has three possible values: 0, 1, and 2. Its assignment depends on whether the application calls the parameter and if the user's current page level matches the assigned value of the page (\textit{val}). If there is no match, the level defaults to 2. As shown in Fig. \ref{Bwapp}.

\textbf{DVWA} is a web application specifically designed to identify security vulnerabilities. In order to enforce strict user verification, the application utilizes the validation function \textit{dvwaIsLoggedIn()} to address any vulnerabilities present in the \textit{VP} \textit{setup.php}. Additionally, the application incorporates the privilege parameter \textit{security} to establish different levels of security on the \textit{VP} \textit{setup.php}. The \textit{security} parameter consists of four possible values: \textit{low, medium, high}, and \textit{impossible}. The assigned security level is determined based on whether the application calls the \textit{security} parameter and if the user's current page level matches the assigned value (\textit{val}) of the page. If there is no match, the highest security level, \textit{high}, is set as the default. As shown in Fig. \ref{DVWA}.

\textbf{Scarf} is a research forum and conference hosted by Stanford. Within this forum, two validation functions, namely \textit{is\_admin()} and \textit{require\_loggedin()}, are utilized, both of which use \textit{privilege} as the privilege parameter. The purpose of the \textit{require\_loggedin()} function is used for strict user verification. On the other hand, the \textit{is\_admin()} function is used to enforce the role's privilege verification, addressing vulnerabilities in the \textit{VPs}: \textit{generaloptions.php, comments.php}, and \textit{showsessions.php}. As shown in Fig.\ref{scarf}.

\textbf{Events lister} is a PHP application developed for managing activities, meetings, schedules, and other events. In order to mitigate vulnerabilities in the \textit{VPs}, specifically \textit{user\_add.php, add\_user.php}, and \textit{setup.php}, the application employs the validation function \textit{checkuser()}. This function relies on the privilege parameter $\$\_SESSION['validUser']$. As shown in Fig. \ref{Events lister}.

\textbf{Mybb} is an open-source forum platform freely available to the public that adheres to standard forum structure and model. To ensure strict user verification, the application utilizes the validation function \textit{user\_permissions()} in conjunction with privilege parameters \textit{uid} and \textit{password}. Additionally, the role's privilege verification is achieved through the validation function \textit{user\_admin\_permissions()}. Finally, the vulnerabilities in \textit{VP} such as \textit{memberlist.php} and \textit{admin/index.php} are repaired. As shown in Fig. \ref{Mybb}.

\textbf{OnlineStore} is an e-commerce platform developed in JAVA and constructed using the MVC framework. The application integrates strict user verification and the role's privilege verification by utilizing the privilege parameter \textit{user}. This parameter effectively repairs vulnerabilities in the \textit{VPs} such as \textit{order?method=findMyOrdersByPage} and \textit{order?method=getProductById}. As shown in Fig. \ref{OnlineStore}.

\textbf{JsForum} is a comprehensive J2EE forum, also known as a bulletin board, developed using the Struts MVC framework. The application emphasizes strict user verification by incorporating privilege parameters such as \textit{sessionUsername} and \textit{sessionPassword}. Additionally, the role's privilege verification is achieved through the utilization of the \textit{sessionType} parameter, which is employed to address and resolve vulnerabilities present in the \textit{VPs}, namely \textit{message.jsp} and \textit{editmessage.jsp}. As shown in Fig. \ref{JsForum}.
% The provided patch code, as illustrated in Figure \ref{JsForum}, focuses on resolving issues in the \textit{editmessage} page by implementing three privilege parameters.



\begin{figure*}[h]
\begin{minipage}[t]{0.48\textwidth}
\begin{algorithm}[H]
\small
\renewcommand{\thealgorithm}{}
\floatname{algorithm}{ }
\caption{\textit{member.php (before patch)}}
\begin{algorithmic}[1]
\State \textless?php
\State \$page = 'member';
\State include ("header.php");
\State include ("includes/window\_top.php");
\State \$gid = \$\_GET['id'];
\State ?\textgreater
\State <title\textgreater \textless?php print \$title; ?\textgreater \textless/title\textgreater 
\State …
\end{algorithmic}
\end{algorithm}
\end{minipage}
\hfill
\begin{minipage}[t]{0.48\textwidth}
\begin{algorithm}[H]
\renewcommand{\thealgorithm}{}
\small
\floatname{algorithm}{ }
\caption{\textit{member.php (after patch)}}
\begin{algorithmic}[1]
\State \textless?php
\State \textcolor[rgb]{0.16,0.32,0.66}{include("patch.php"); }
\State \textcolor[rgb]{0.16,0.32,0.66}{repair();}
\State \$page = 'member';
\State include ("header.php");
\State include ("includes/window\_top.php");
\State \$gid = \$\_GET['id'];
\State ?\textgreater 
\State \textless title\textgreater \textless?php print \$title; ?\textgreater \textless/title\textgreater 
\State …
\end{algorithmic}
\end{algorithm}
\end{minipage}
\hfill
\begin{minipage}[t]{1\textwidth}
\begin{algorithm}[H]
\renewcommand{\thealgorithm}{}
\small
\floatname{algorithm}{ }
\caption{\textit{patch.php (containing the patch code)}}
\begin{algorithmic}[1]
\State \textless?php
\State function repair ()\{
\Statex \textcolor[rgb]{0.16,0.32,0.66}{// Implement strict user verification by employing the privilege parameters \$\_SESSION['username']) to mitigate HPE vulnerabilities, as detailed in section \ref{Merging Privilege Parameters and Validation Functions}. Ensure that these parameters are set before proceeding.}
\If{(! isset(\$\_SESSION['username']))}\{ 
    \State header('location:./login.php');\}
\EndIf
\Statex \textcolor[rgb]{0.16,0.32,0.66}{// Implement role's privilege verification by employing the privilege parameter \$\_SESSION['awcm\_cp']) to mitigate VPE vulnerabilities. Ensure that these parameters are set before proceeding.}
\If{(isset(\$\_SESSION['awcm\_cp']))}\{ 
    \If{(\$\_SESSION['awcm\_cp'] == 'no')}
        \State \{header("location:./index.php");\}
    \EndIf
\EndIf
\State \}\}
\State?\textgreater 
\end{algorithmic}
\end{algorithm}
\end{minipage}%
\hfill
\caption{Access control patch code for AWCMs application in PHP using privilege parameters.}
\label{AWCMs}
\end{figure*}

\begin{figure*}[h]
\begin{minipage}[t]{0.48\textwidth}
\begin{algorithm}[H]
\renewcommand{\thealgorithm}{}
\small
\floatname{algorithm}{ }
\caption{\textit{modifica\_votanti.php (before patch)}}
\begin{algorithmic}[1]
\State \textless html\textgreater 
\State \textless head\textgreater 
    \State \textless title\textgreater PHPOLL\textless/title\textgreater 
    \State \textless link rel="stylesheet" href="../css/phpoll\_layout.css" title="phpoll layout" /\textgreater 
\State \textless /head\textgreater 
\State \textless /html\textgreater 
\State \textless ?php
\State include "../config/config.php";
\State ...
\State ?\textgreater 
\end{algorithmic}
\end{algorithm}
\end{minipage}
\hfill
\begin{minipage}[t]{0.48\textwidth}
\begin{algorithm}[H]
\renewcommand{\thealgorithm}{}
\small
\floatname{algorithm}{ }
\caption{\textit{modifica\_votanti.php (after patch)}}
\begin{algorithmic}[1]
\State \textless html\textgreater 
\State \textless head\textgreater 
    \State \textless title\textgreater PHPOLL\textless/title\textgreater 
    \State \textless link rel="stylesheet" href="../css/phpoll\_layout.css" title="phpoll layout" /\textgreater 
\State \textless/head\textgreater 
\State \textless/html\textgreater 
\State \textless?php
\State \textcolor[rgb]{0.16,0.32,0.66}{include("patch.php");}
\State \textcolor[rgb]{0.16,0.32,0.66}{repair();}
\State include "../config/config.php";
\State ...
\State ?\textgreater 
\end{algorithmic}
\end{algorithm}
\end{minipage}
\hfill
\begin{minipage}[t]{1\textwidth}
\begin{algorithm}[H]
\renewcommand{\thealgorithm}{}
\small
\floatname{algorithm}{ }
\caption{\textit{patch.php (containing the patch code)}}
\begin{algorithmic}[1]
\State \textless?php
\State function repair ()\{
 \Statex \textcolor[rgb]{0.16,0.32,0.66}{// Implement strict user verification by employing the privilege parameters \$\_COOKIE[\$string\_cook\_login] to mitigate HPE vulnerabilities, and implement role's privilege verification by employing the privilege parameter \$\_COOKIE[\$string\_cook\_password] to mitigate VPE vulnerabilities. As detailed in section \ref{Merging Privilege Parameters and Validation Functions}. Ensure that these parameters are set before proceeding.}
\If {(!(isset(\$\_COOKIE[\$string\_cook\_login]) \&\& isset(\$\_COOKIE[\$string\_cook\_password])))}\{
    \State header('location:./index');
\EndIf
\State \}\}
\State?\textgreater 
\end{algorithmic}
\end{algorithm}
\end{minipage}%
\hfill
\caption{Access control patch code for Phpoll application in PHP using privilege parameters.}
\label{Phpoll}
\end{figure*}

\begin{figure*}[h]
\begin{minipage}[t]{0.48\textwidth}
\begin{algorithm}[H]
\small
\renewcommand{\thealgorithm}{}
\floatname{algorithm}{ }
\caption{\textit{backdoor.php (before patch)}}
\begin{algorithmic}[1]
\State \textless?php
\If{(isset(\$\_REQUEST["upload"]))}\{
     \State \$dir=\$\_REQUEST["uploadDir"];\}
\If{(phpversion()\textless'4.1.0')}\{
     \State \$file=\$HTTP\_POST\_FILES["file"]["name"];\} 
     \Statex @move\_uploaded\_file(\$HTTP\_POST\_FILES["file"]["tmp\_name"]);
     \EndIf
\EndIf
\State ...
\State ?\textgreater 
\end{algorithmic}
\end{algorithm}
\end{minipage}
\hfill
\begin{minipage}[t]{0.48\textwidth}
\begin{algorithm}[H]
\renewcommand{\thealgorithm}{}
\small
\floatname{algorithm}{ }
\caption{\textit{backdoor.php (after patch)}}
\begin{algorithmic}[1]
\State \textless?php
\State \textcolor[rgb]{0.16,0.32,0.66}{include("patch.php");}
\State \textcolor[rgb]{0.16,0.32,0.66}{repair();}
\If{(isset(\$\_REQUEST["upload"]))}\{
     \State \$dir=\$\_REQUEST["uploadDir"];\}
\If{(phpversion()\textless'4.1.0')}\{
     \State \$file=\$HTTP\_POST\_FILES["file"]["name"];\}
     \Statex @move\_uploaded\_file(\$HTTP\_POST\_FILES["file"]["tmp\_name"]);
     \EndIf
\EndIf
\State ...
\State ?\textgreater 
\end{algorithmic}
\end{algorithm}
\end{minipage}
\hfill
\begin{minipage}[t]{1\textwidth}
\begin{algorithm}[H]
\small
\renewcommand{\thealgorithm}{}
\floatname{algorithm}{ }
\caption{\textit{patch.php (containing the patch code)}}
\begin{algorithmic}[1]
\State \textless?php
\State function repair ()\{
    \Statex \textcolor[rgb]{0.16,0.32,0.66}{// Implement strict user verification by employing the privilege parameters \$username and \$password to mitigate HPE vulnerabilities, as detailed in section \ref{Merging Privilege Parameters and Validation Functions}. Ensure that these parameters are set before proceeding.}
    \If {(!(isset(\$username) \&\& isset(\$password)))}\{
            \State header('location:./index');\}
    \EndIf
         \Statex \textcolor[rgb]{0.16,0.32,0.66}{// Implement role's privilege verification by employing the privilege parameter \$\_COOKIE['security\_level']to mitigate VPE vulnerabilities. Ensure that this parameters are set before proceeding.}
        \If{(isset(\$\_COOKIE['security\_level']))}\{
        \If{(\$\_COOKIE['security\_level'] != get\_current\_user(\$user).level)}\{
            \State reset('security\_level','2'); \}
        \EndIf
    \EndIf
\State \}\}
\State?\textgreater 
\end{algorithmic}
\end{algorithm}
\end{minipage}
\caption{Access control patch code for Bwapp application in PHP using privilege parameters.}
\label{Bwapp}
\end{figure*}

\begin{figure*}[ht]
\begin{minipage}[t]{0.48\textwidth}
\begin{algorithm}[H]
\small
\renewcommand{\thealgorithm}{}
\floatname{algorithm}{ }
\caption{\textit{setup.php (before patch})}
\begin{algorithmic}[1]
\State \textless?php
\State define( 'DVWA\_WEB\_PAGE\_TO\_ROOT', '' );
\State require\_once DVWA\_WEB\_PAGE\_TO\_ROOT . 'dvwa/includes/dvwaPage.inc.php';
\State dvwaPageStartup( array( 'phpids' ) );
\State \$page = dvwaPageNewGrab();
\State \$page[ 'title' ]   = 'Setup' . \$page[ 'title\_separator' ].\$page[ 'title' ];
\State \$page[ 'page\_id' ] = 'setup';
\State ...
\State ?\textgreater 
\end{algorithmic}
\end{algorithm}
\end{minipage}
\hfill
\begin{minipage}[t]{0.48\textwidth}
\begin{algorithm}[H]
\small
\renewcommand{\thealgorithm}{}
\floatname{algorithm}{ }
\caption{\textit{setup.php (after patch)}}
\begin{algorithmic}[1]
\State \textless?php
\State \textcolor[rgb]{0.16,0.32,0.66}{require\_once 'patch.php';}
\State \textcolor[rgb]{0.16,0.32,0.66}{repair();}
\State define( 'DVWA\_WEB\_PAGE\_TO\_ROOT', '' );
\State require\_once DVWA\_WEB\_PAGE\_TO\_ROOT . 'dvwa/includes/dvwaPage.inc.php';
\State dvwaPageStartup( array( 'phpids' ) );
\State \$page = dvwaPageNewGrab();
\State \$page[ 'title' ]   = 'Setup' . \$page[ 'title\_separator' ].\$page[ 'title' ];
\State \$page[ 'page\_id' ] = 'setup';
\State ...
\State ?\textgreater 
\end{algorithmic}
\end{algorithm}
\end{minipage}
\hfill
\begin{minipage}[t]{1\textwidth}
\begin{algorithm}[H]
\small
\renewcommand{\thealgorithm}{}
\floatname{algorithm}{ }
\caption{\textit{patch.php (containing the patch code)}}
\begin{algorithmic}[1]
\State \textless?php
\State function repair ()\{
\Statex \textcolor[rgb]{0.16,0.32,0.66}{// Implement strict user verification by calling the location of the validation function exists, as detailed in section \ref{Merging Privilege Parameters and Validation Functions}.}
\State include('includes/function.php');
\Statex \textcolor[rgb]{0.16,0.32,0.66}{// Employ the validation function dvwaIsLoggedIn() to mitigate HPE vulnerabilities.}
\State dvwaIsLoggedIn();
 \Statex \textcolor[rgb]{0.16,0.32,0.66}{// Implement role's privilege verification by employing the privilege parameter \$\_COOKIE['security'] to mitigate VPE vulnerabilities. Ensure that this parameters are set before proceeding.}
\If{(isset(\$\_COOKIE['security']))}\{ 
    \If{(\$\_COOKIE['security'] != get\_current\_user(\$user).level)}\{
        \State reset('security', 'high');
    \EndIf
\EndIf
\State \}\}
\State?\textgreater 
\end{algorithmic}
\end{algorithm}
\end{minipage}%
\caption{Access control patch code for DVWA application in PHP using validation function and privilege parameter.}
\label{DVWA}
\end{figure*}

\begin{figure*}[ht]
\begin{minipage}[t]{0.48\textwidth}
\begin{algorithm}[H]
\small
\renewcommand{\thealgorithm}{}
\floatname{algorithm}{ }
\caption{\textit{generaloptions.php (before patch)}}
\begin{algorithmic}[1]
\State \textless?php include\_once("header.php");
\If{(isset(\$\_POST['submit']))} \{
    \For{ foreach (\$\_POST as \$name =\textgreater  \$value)} \{
        \State \$name = mysql\_real\_escape\_string(\$name);
        \State \$name = str\_replace("\_", " ", \$name);
        \State \$value = mysql\_real\_escape\_string(\$value);
        \State query("UPDATE options SET value='\$value' WHERE name='\$name'");
    \EndFor
\EndIf
\State \}\}
\State …
\State ?\textgreater 
\end{algorithmic}
\end{algorithm}
\end{minipage}
\hfill
\begin{minipage}[t]{0.48\textwidth}
\begin{algorithm}[H]
\small
\renewcommand{\thealgorithm}{}
\floatname{algorithm}{ }
\caption{\textit{generaloptions.php (after patch)}}
\begin{algorithmic}[1]
\State \textless?php include\_once("header.php");
\State \textcolor[rgb]{0.16,0.32,0.66}{include\_once(“patch.php");}
\State \textcolor[rgb]{0.16,0.32,0.66}{repair();}
\If{(isset(\$\_POST['submit']))} \{
    \For{ foreach (\$\_POST as \$name =\textgreater  \$value)} \{
        \State \$name = mysql\_real\_escape\_string(\$name);
        \State \$name = str\_replace("\_", " ", \$name);
        \State \$value = mysql\_real\_escape\_string(\$value);
        \State query("UPDATE options SET value='\$value' WHERE name='\$name'");
    \EndFor
\EndIf
\State \}\}
\State …
\State ?\textgreater 
\end{algorithmic}
\end{algorithm}
\end{minipage}
\hfill
\begin{minipage}[t]{1\textwidth}
\begin{algorithm}[H]
\small
\renewcommand{\thealgorithm}{}
\floatname{algorithm}{ }
\caption{\textit{patch.php (containing the patch code)}}
\begin{algorithmic}[1]
\State \textless?php
\State function repair ()\{
\Statex \textcolor[rgb]{0.16,0.32,0.66}{// Call the location of the validation function exists, as detailed in section \ref{Merging Privilege Parameters and Validation Functions}.}
\State include "...\textbackslash scarf \textbackslash \textbackslash functions.php";
\Statex \textcolor[rgb]{0.16,0.32,0.66}{// Implement strict user verification by employing the validation function require\_loggedin() to mitigate HPE vulnerabilities.}
\State require\_loggedin();
\Statex \textcolor[rgb]{0.16,0.32,0.66}{// Implement role's privilege verification by employing the validation function is\_admin() to mitigate VPE vulnerabilities.}
\State is\_admin();
\State?\textgreater 
\end{algorithmic}
\end{algorithm}
\end{minipage}
\caption{Access control patch code for Scarf application in PHP using validation functions.}
\label{scarf}
\end{figure*}

\begin{figure*}[ht]
\begin{minipage}[t]{0.48\textwidth}
\begin{algorithm}[H]
\small
\renewcommand{\thealgorithm}{}
\small
\floatname{algorithm}{ }
\caption{\textit{user\_add.php (before patch)}}
\begin{algorithmic}[1]
\State \textless?php
\State ?\textgreater 
\State \textless html xmlns="http://www.w3.org/1999/xhtml"\textgreater 
\State \textless head\textgreater 
\State \textless title\textgreater Add a user\textless/title\textgreater 
\State \textless/head\textgreater 
\State \textless body\textgreater 
\State \textless h1\textgreater Add User\textless/h1\textgreater 
\State ...
\State \textless/body\textgreater 
\State \textless/html\textgreater 
\end{algorithmic}
\end{algorithm}
\end{minipage}
\hfill
\begin{minipage}[t]{0.48\textwidth}
\begin{algorithm}[H]
\renewcommand{\thealgorithm}{}
\small
\floatname{algorithm}{ }
\caption{\textit{user\_add.php (after patch)}}
\begin{algorithmic}[1]
\State \textless?php
\State \textcolor[rgb]{0.16,0.32,0.66}{include("patch.php");}
\State \textcolor[rgb]{0.16,0.32,0.66}{repair();}
\State ?\textgreater 
\State \textless html xmlns="http://www.w3.org/1999/xhtml"\textgreater 
\State \textless head\textgreater 
\State \textless title\textgreater Add a user\textless/title\textgreater 
\State \textless/head\textgreater 
\State \textless body\textgreater 
\State \textless h1\textgreater Add User\textless/h1\textgreater 
\State ...
\State \textless/body\textgreater 
\State \textless/html\textgreater 
\end{algorithmic}
\end{algorithm}
\end{minipage}
\hfill
\begin{minipage}[t]{1\textwidth}
\begin{algorithm}[H]
\small
\renewcommand{\thealgorithm}{}
\small
\floatname{algorithm}{ }
\caption{\textit{patch.php (containing the patch code)}}
\begin{algorithmic}[1]
\State \textless?php
\State function repair () \{
\Statex \textcolor[rgb]{0.16,0.32,0.66}{// Call the location of the validation function exists, as detailed in section \ref{Merging Privilege Parameters and Validation Functions}.}
\State include('functions.php');
\Statex \textcolor[rgb]{0.16,0.32,0.66}{// Implement strict user verification and role's privilege verification by employing the validation function checkuser() to mitigate HPE and VPE vulnerabilities.}
\State checkuser(); 
\State?\textgreater 
\end{algorithmic}
\end{algorithm}
\end{minipage}%
\caption{Access control patch code for Events lister application in PHP using validation function.}
\label{Events lister}
\end{figure*}

\begin{figure*}[h]
\begin{minipage}[t]{0.48\textwidth}
\begin{algorithm}[H]
\renewcommand{\thealgorithm}{}
\small
\floatname{algorithm}{ }
\caption{\textit{memberlist.php (before patch)}}
\begin{algorithmic}[1]
\State \textless?php
\State define("IN\_MYBB", 1);
\State define("IGNORE\_CLEAN\_VARS", "sid");
\State define('THIS\_SCRIPT', 'member.php');
\State define("ALLOWABLE\_PAGE","register,do\_register,login,
do\_login,logout,lostpw,do\_lostpw,activate,resendactivation,
do\_resendactivation,resetpassword,viewnotes");
\State \$nosession['avatar'] = 1;
\State ...
\State ?\textgreater 
\end{algorithmic}
\end{algorithm}
\end{minipage}
\hfill
\begin{minipage}[t]{0.48\textwidth}
\begin{algorithm}[H]
\small
\renewcommand{\thealgorithm}{}
\floatname{algorithm}{ }
\caption{\textit{memberlist.php (after patch)}}
\begin{algorithmic}[1]
\State \textless?php
\State \textcolor[rgb]{0.16,0.32,0.66}{include("patch.php");}
\State \textcolor[rgb]{0.16,0.32,0.66}{repair();}
\State define("IN\_MYBB", 1);
\State define("IGNORE\_CLEAN\_VARS", "sid");
\State define('THIS\_SCRIPT', 'member.php');
\State define("ALLOWABLE\_PAGE","register,do\_register,login,
do\_login,logout,lostpw,do\_lostpw,activate,
resendactivation,do\_resendactivation,resetpassword,viewnotes");
\State \$nosession['avatar'] = 1;
\State ...
\State ?\textgreater 
\end{algorithmic}
\end{algorithm}
\end{minipage}
\hfill
\begin{minipage}[t]{1\textwidth}
\begin{algorithm}[H]
\small
\renewcommand{\thealgorithm}{}
\floatname{algorithm}{ }
\caption{\textit{patch.php (containing the patch code)}}
\begin{algorithmic}[1]
\State \textless?php 
\State function repair()\{
\Statex \textcolor[rgb]{0.16,0.32,0.66}{// Call the location of the validation function exists, as detailed in section \ref{Merging Privilege Parameters and Validation Functions}.}
\State include 'Upload/inc/functions.php';
\Statex \textcolor[rgb]{0.16,0.32,0.66}{// Implement strict user verification by employing the validation function validate\_password\_from\_uid() to mitigate HPE vulnerabilities.}
\State validate\_password\_from\_uid();
\Statex \textcolor[rgb]{0.16,0.32,0.66}{// Implement role's privilege verification by employing the validation function user\_admin\_privilege() to mitigate VPE vulnerabilities.}
\State user\_admin\_privilege();
\State \}
\State ?\textgreater 
\end{algorithmic}
\end{algorithm}
\end{minipage}%
\caption{Access control patch code for Mybb application in PHP using validation functions.}
\label{Mybb}
\end{figure*}

\begin{figure*}[h]
\begin{minipage}[t]{0.48\textwidth}
\begin{algorithm}[H]
\small
\renewcommand{\thealgorithm}{}
\floatname{algorithm}{ }
\caption{\textit{OrderServlet (before patch)}}
\begin{algorithmic}[1]
\State public String getProductById(HttpServletRequest request, HttpServletResponse response) throws ServletException, IOException \{
\State try \{
\State String sOrderId = request.getParameter("orderId");
\State Order order = this.orderService.getOrderById(sOrderId);
\State request.setAttribute("order", order);
\State     return "/jsp/order\_info.jsp";
\State \} catch (Exception var5) \{
\State var5.printStackTrace();
\State request.setAttribute("sMessage", "defeat");
\State return "/jsp/message.jsp";
\State \}\}
\end{algorithmic}
\end{algorithm}
\end{minipage}
\hfill
\begin{minipage}[t]{0.48\textwidth}
\begin{algorithm}[H]
\small
\renewcommand{\thealgorithm}{}
\floatname{algorithm}{ }
\caption{\textit{OrderServlet (after patch)}}
\begin{algorithmic}[1]
\State public String getProductById(HttpServletRequest request, HttpServletResponse response) throws ServletException, IOException \{
\State try \{
\State \textcolor[rgb]{0.16,0.32,0.66}{patch obj = new patch();}
\State \textcolor[rgb]{0.16,0.32,0.66}{obj.repair();}
\State String sOrderId = request.getParameter("orderId");
\State Order order = this.orderService.getOrderById(sOrderId);
\State request.setAttribute("order", order);
\State     return "/jsp/order\_info.jsp";
\State \} catch (Exception var5) \{
\State var5.printStackTrace();
\State request.setAttribute("sMessage", "defeat");
\State return "/jsp/message.jsp";
\State \}\}
\end{algorithmic}
\end{algorithm}
\end{minipage}
\hfill
\begin{minipage}[t]{1\textwidth}
\begin{algorithm}[H]
\small
\renewcommand{\thealgorithm}{}
\floatname{algorithm}{ }
\caption{\textit{patch.php (containing the patch code)}}
\begin{algorithmic}[1]
\State public class patch \{
\State public String String repair()\{
\Statex \textcolor[rgb]{0.16,0.32,0.66}{// Implement strict user verification and role's privilege verification by employing the privilege parameters "user" to mitigate HPE and VPE vulnerabilities, as detailed in section \ref{Merging Privilege Parameters and Validation Functions}. Ensure that these parameters are set before proceeding.}
\State User user = (User)request.getSession().getAttribute("user");
\If{(!user)}\{ 
    \State request.setAttribute("sMessage", "login again");
    \State return "/jsp/message.jsp";\}
\EndIf
\State \}\}
\end{algorithmic}
\end{algorithm}
\end{minipage}%
\caption{Access control patch code for OnlineStore application in JAVA using privilege parameters.}
\label{OnlineStore}
\end{figure*}

\begin{figure*}
\begin{minipage}[t]{0.48\textwidth}
\begin{algorithm}[H]
\small
\renewcommand{\thealgorithm}{}
\floatname{algorithm}{ }
\caption{\textit{editmessage.jsp (before patch)}}
\begin{algorithmic}[1]
\State \textless\% String requestPage = request.getParameter("page"); \%\textgreater 
\State \textless\%
\If{(requestPage == null)}\{
   \State requestPage = "0";
\State \}\%\textgreater 
\EndIf
\State \textless\%  
\If{(requestPage.equals("thread"))}\{ \%\textgreater 
   \State \textless\% String forum\_id = request.getParameter("forum\_id"); \%\textgreater 
   \State \textless jsp:include page="thread.jsp" flush="true" \textgreater 
      \State \textless jsp:param name="forum\_id" value="\textless\%= forum\_id \%\textgreater " /\textgreater 
   \State \textless/jsp:include\textgreater 
\EndIf
\State ...
\State \}\%\textgreater 
\end{algorithmic}
\end{algorithm}
\end{minipage}
\hfill
\begin{minipage}[t]{0.48\textwidth}
\begin{algorithm}[H]
\small
\renewcommand{\thealgorithm}{}
\floatname{algorithm}{ }
\caption{\textit{editmessage.jsp (after patch)}}
\begin{algorithmic}[1]
\State \textcolor[rgb]{0.16,0.32,0.66}{\textless\%@ page import="patch" \%\textgreater }
\State \textcolor[rgb]{0.16,0.32,0.66}{\textless\% repair();\%\textgreater }
\State \textless\% String requestPage = request.getParameter("page"); \%\textgreater 
\State \textless\%
\If{(requestPage == null)}\{
   \State requestPage = "0";
\State \}\%\textgreater 
\EndIf
\State \textless\%  
\If{(requestPage.equals("thread"))}\{ \%\textgreater 
   \State \textless\% String forum\_id = request.getParameter("forum\_id"); \%\textgreater 
   \State \textless jsp:include page="thread.jsp" flush="true" \textgreater 
      \State \textless jsp:param name="forum\_id" value="\textless\%= forum\_id \%\textgreater " /\textgreater 
   \State \textless/jsp:include\textgreater 
\EndIf
\State ...
\State \}\%\textgreater 
\end{algorithmic}
\end{algorithm}
\end{minipage}
\hfill
\begin{minipage}[t]{1\textwidth}
\begin{algorithm}[H]
\small
\renewcommand{\thealgorithm}{}
\floatname{algorithm}{ }
\caption{\textit{patch.php (containing the patch code)}}
\begin{algorithmic}[1]
\State \textless\%
\State String repair() \{
    \Statex \textcolor[rgb]{0.16,0.32,0.66}{// Implement strict user verification by employing the privilege parameters "sessionUsername" and "sessionPassword" to mitigate HPE vulnerabilities, as detailed in section \ref{Merging Privilege Parameters and Validation Functions}. Ensure that these parameters are set before proceeding.}
    \If{(!(session.getAttribute("sessionUsername") != null \&\& session.getAttribute("sessionPassword") != null))}\{
    \State \textless a href="./register.jsp"\textgreater Register\textless/a\textgreater 
    \State \}
    \EndIf
    \Statex \textcolor[rgb]{0.16,0.32,0.66}{// Implement role's privilege verification by employing the privilege parameter "sessionType" to mitigate VPE vulnerabilities. Ensure that these parameters are set before proceeding.}
    \If{(isset(session.getAttribute("sessionType"))}\{
    \State \textless\% session.getAttribute("sessionType").equals("Admin"))\{ \%\textgreater 
        \State \textless jsp:include page="./include/table\_title.jsp" flush="true"\textgreater 
        \State \textless jsp:param name="title" value="index" /\textgreater 
        \State \textless jsp:param name="colspan" value="1" /\textgreater 
        \State \textless jsp:param name="align" value="middle" /\textgreater 
        \State\textless/jsp:include\textgreater 
    \State \textless\% \} \%\textgreater 
    \State \}
    \EndIf
\State \}\%\textgreater

\end{algorithmic}
\end{algorithm}
\end{minipage}%
\caption{Access control patch code for JsForum application in JAVA using privilege parameters.}
\label{JsForum}
\end{figure*}

\end{CJK}
\end{document}
